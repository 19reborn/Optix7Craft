% Copyright 2021 wr786
% Permission is granted to copy, distribute and/or modify this
% document under the terms of the Creative Commons
% Attribution 4.0 International (CC BY 4.0)
% http://creativecommons.org/licenses/by/4.0/

\documentclass{article}
\usepackage{ctex}
\usepackage{amsmath}

\title{OptiX7Craft - 基于OptiX7的光线追踪仿Minecraft游戏}
\author{王益明 1900012416; 王锐 1900011029; 熊穗宁 1900010694}

\begin{document}
    \maketitle
    \pagestyle{plain}

    \section{项目简介}
    \subsection{背景}
    % 这里都是抄的SekiroSoul组的
    NVIDIA OptiX Ray Tracking Engine (下称OptiX) 是为 NVIDIA GPU 等高度并行架构设计的可编程系统。OptiX
引擎基于一个关键的观察,即大多数射线跟踪算法都可以使用一组可编程操作来实现。因此,OptiX 的
核心是一个特定领域的实时编译器,它通过组合用户提供的用于射线生成、材质着色、对象相交和场景
遍历的程序来生成定制的射线跟踪内核。这使得实现一组高度多样化的基于光线跟踪的算法和应用程序
成为可能,包括交互式呈现、脱机呈现、碰撞检测系统、人工智能查询和科学模拟 (如声音传播)。
    而OptiX7,是OptiX最新的一个版本。
    
    \subsection{应用}
    本项目使用OptiX7光线追踪引擎进行游戏场景渲染,实现模拟现实中的反射、折射、光照、阴影等效果。
    需要指出的是,本项目也是一个可以游玩的游戏,仿制Minecraft。
    % 这里应该有张印象图

    \section{理论}


    \section{技术细节}

    \section{实现效果}
    % 这里应该有一堆图



    \textbf{注:}代码未来将开源在https://github.com/19reborn/Ray-Tracing-Project

\end{document}